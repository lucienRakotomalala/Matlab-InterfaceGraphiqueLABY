\doublebox{
\begin{minipage}{.9\textwidth}
\vspace{1mm}
\underline{\textbf{\Large  Problematic}}\\
\label{chap:Intro}
\hspace{3mm}  \begin{minipage}{.96\textwidth}
\vspace{1mm}\hspace{2mm} The project focuses on the modeling and analysis with Discrete Events Systems (DES) of a system composed of a dynamic labyrinth and two objects moving inside the labyrinth.\\ This analysis will be used to identify properties to test this uncertain model.
We will contextualize our project with current knowledge in DES modeling.\\
First, we will redefine the definition of what we will call a "labyrinth" and then we will recall the formal definitions of DES \cite{controlOfDiscreteEvent_Ramadge} \cite{DiagnostabilityOfDES}.\\
In a second step, we will discuss more precisely the properties and products that we will use for the analysis and verification of our models. We will use the formal definitions explained in these books \cite{cassandras2009introduction} \cite{introductionAutomataTheoryLangageComputation_2007}. \\
The last part will remind us the scientific approach chosen and the types of verification and validation that will allow us to test our prototypes.
\end{minipage}

% Je dois contenir : 
% Problématiques (Comment peut on modéliser un labyrinthe ...

% Livre de départ : 
%%%%  Introduction to Discrete Event Systems :: Cassandras
% --> Infos utiles : 
% --> COmment l'a ton trouvé

%%%% An epistemic and dynamic approach to abductive reasoning: Abductive problem and abductive solution :: 

%%%% Stateless techniques for generating global and local test oracles for message-passing concurrent programs
\end{minipage}
}