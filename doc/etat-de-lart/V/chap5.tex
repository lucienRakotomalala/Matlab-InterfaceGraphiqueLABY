\chapter{Verification and validation}

\section{Approach methods}

%A comprendre
%We chose to study three principal types of approaches. They are adapted according to the research carried out (fundamental or applied). This part consists of presenting these types of reasoning.
We chose to study three types of approaches. There are adapted for researches and development fundamental or applied. During our project we shall use the next three approaches by using theoretical proofs or validation tests. This part consists of introducing these different types of reasoning.
% Nous avons choisi d'étudier trois types principaux d'approches. Ils sont adaptés en fonction des recherches effectuées (fondamentales ou appliquées).Durant notre projet, nous aurons l'occasion de les utiliser en trouvant des preuves théoriques ou en réalisant des test de validation. Cette partie consiste à présenter ces types de raisonnement.

\subsection{Inductive Approach}
%An inductive approach begins with observation and then generalizes to broader theories \cite{o2004essential}.
%This reasoning is divided in four steps, it is necessary to begin with the observations for then analyze these observations. For finally generalize in the form of hypotheses and verify them.
An inductive approach begins by observing of our system used to generalize our theories.[Wilson, J. (2010) “Essentials of Business Research: A Guide to Doing Your Research Project” SAGE Publications, p7] \cite{buisnessREsearchWilson}\\
This reasoning is composed of four steps :
\begin{itemize}
\item Observing
\item Analyzing the observations
\item Generalizing thanks to hypothesis
\item Verifying hypothesis
\end{itemize}
% Une approche inductive commence par l’observation pour ensuite généraliser sur des théories plus large [Wilson, J. (2010) “Essentials of Business Research: A Guide to Doing Your Research Project” SAGE Publications, p7]. 
% Ce raisonnement est divisé en quatre étapes, il est faut commencer par des observations pour ensuite analyser ces observations. Pour enfin généraliser sous la forme d'hypothèses et de les vérifier.

\subsection{Deductive Approach}
%Deductive reasoning involves testing a theory and seeking confirmation through observations\cite{o2004essential}. We can thus prove and explain implicit relationships of cause and effect.
%The deductive approach generally takes place in four stages, at first it must deduce hypotheses from the theory, for then test these hypotheses and analyze the results. Depending on these results, the theory can be modified.
The deductive reasoning consists in testing a theory then confirming if throw the analysis of the observations  [Wilson, J. (2010) “Essentials of Business Research: A Guide to Doing Your Research Project” SAGE Publications, p5] \cite{buisnessREsearchWilson}\\
We can prove and explain implied a cause to effect relationships.
This reasoning is composed of four steps :
\begin{itemize}
\item Deducing hypothesis from theory
\item Testing hypothesis
\item Analyzing results %sous forme d'hypothèse
\item Modifying theory if necessary
\end{itemize}
% Le raisonnement déductif consiste à tester une théorie et à rechercher une confirmation par des observations  [Wilson, J. (2010) “Essentials of Business Research: A Guide to Doing Your Research Project” SAGE Publications, p5]. On peut ainsi prouver et expliquer des relations implicite de cause à effet. 
% L'approche déductive se déroule généralement en quatre étapes, pour commencer, il faut déduire des hypothèses de la théorie, puis tester ces hypothèses et analyser les résultats. En fonction de ces résultats, la théorie peut être modifiée.

\subsection{Abductive Approach}
%The abductive reasoning is to give explanations to an experiment and to obtain a hypothesis. This reasoning is often used in the areas of scientific discovery, legal reasoning and diagnosis because it is said to be conducive to innovation. Overall, abductive reasoning can be viewed as a process of four phases:
%First, we must recognize the existence of an abductive problem. Secondly, candidate solutions must be identified. Third, choose the most plausible solutions. Fourth, assimilate the solutions chosen \cite{VELAZQUEZQUESADA2013505}.
The abductiv reasoning consists in finding explanations of an experiment and hypothesis. It is used for scientific discoveries because it is more convenient to innovative approaches. As explained by Wilson, the abductiv approach contains four steps :
\begin{itemize}
\item Recognizing the existence of an abductiv problem
\item Identifying solutions
\item Finding possible solutions
\item Explaining solutions and implementing them to solve the problem %eexpliquer ces solutions et savoir les mettre en œuvre pour résoudre le problème rencontré
\end{itemize}

% Le raisonnement abductif consiste à donner des explications à une expérimentation et à obtenir une hypothèse. Ce raisonnement est souvent utilisé dans le domaine de la découverte scientifique car il est particulièrement adapté à des approches  innovantes. En référence à l’article cité, le raisonnement abductif peut être perçu comme un processus contenant quatre phases :
% Premièrement, il faut reconnaitre l’existence d’un problème abductif. Deuxièmement, il faut identifier les solutions candidates. Troisièmement choisir les solutions les plus plausibles. Quatrièmement, assimiler les solutions retenues.

\section{Test method}
% Afin de valider le code, la mise en place d'une méthode de vérification et de validation du code est nécessaire, il est notamment possible d'utiliser des oracles . Ces oracles permettent de verifier la réussite ou l'echec d'un test. C’est à dire de déterminer si le résultat obtenu à l'issu du test correspond bien au resultat attendu \cite{CARVER2018237}.\\
% Plusieurs cas de test doivent être abordés, en fonction des scénarios désirés. Ces cas de test correspondent à des données de test à mettre en place en entrée de l'oracle. En sortie de ce dernier, le resultat confirmera la validitée ou non du code testé. Ces cas de tests peuvent avoir une approche soit inductive en réalisant une multitude de tests soit abductive en réalisant des test précis et ciblés pour chaque cas critique.

%%%% DeepL and Johanne translation ;) %%%%%%%%%%%%%%%%%%%%%%%
%verifier comment on traduit validation et abductif
In order to validate the code, it is necessary to set up a method of verification and validation of the code, it is notably possible to use \textit{oracles}. An oracle is a small code that runs along the code. It is used to verify the success or failure of a test. It means, we should determine if the result obtained at the end of the test corresponds to the expected one \cite{CARVER2018237}.
Several types of tests are needed, it depends on an expected scenarios. These types of tests correspond to data tests that must be sent at the oracle input. At the output of the oracle, the result will confirm the validity or invalidity of the tested code. These test cases can either have an inductive approach by performing a multiply tests or an abductive approach by performing precise and targeted tests for each critical case.





